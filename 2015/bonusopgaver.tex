\section{\huge{Bonusopgaver}}

\emph{I år er bonusopgaverne sat et niveau op i sværhedsgrad.  Til gengæld
  værdsættes kreative løsninger endnu mere end normalt.}

\subsection{Opgave 0}

Til et instruktormøde i PoP blev der af uransagelige årsager drukket snaps.  Den
udleverede F\#-kode for fakultetsfunktionen er i overensstemmelse med denne
hændelses effekter:

\tiny
\begin{verbatim}
let fac n =
    let mutable f = (n + 1) / 2
    for i in seq { 1 .. (n - 1) / 2 } do
        f <- f * ((n + 1) / 2) * ((n + 1) / 2) - f * i * i
    System.Convert.ToInt32((n = 0)) + ((1 - n % 2) * n + n % 2) * f
[<EntryPoint>]
let main [|s|] =
    printfn "%d! = %d" (System.Int32.Parse s) (fac (System.Int32.Parse s))
    0
\end{verbatim}
\normalsize

Til trods for den anderledes fakultetsfunktionsalgoritme, garanterer en
nissehuebærende kursusansvarlig at koden skam er korrekt.

\subsubsection{(a)}

Du læser i kursusdatabasen at en ønsket PoP-færdighed er ``at kunne lave mindre
programmer [...] med overholdelse af god programmeringsskik og -stil'' og indser
at kodeskik er meget, meget, \emph{meget} vigtigt!

Din opgave er derfor at rydde op i koden, så instruktorerne slipper for at rette
deres egen kode.  Gør det pænt og overskueligt!


\subsubsection{(b)}

Bevis at den givne fakultetsfunktionsalgoritme er korrekt.  \emph{Vink}: Prøv at
håndkøre \texttt{fac} for små værdier af \texttt{n}.


\newpage
\subsection{Opgave 1}

INDSÆT OLEKS' ARK-OPGAVE HER?  Ellers finder jeg ud af noget.


\newpage
\subsection{Opgave 2}

Din instruktor i Oversættere fik besked på at lave en lille regex-opgave.
Instruktoren begyndte frejdigt, men efter at have drukket fem af Kantinens lækre
nye 7,1\% ciders i et forgæves forsøg på at gøre opgaven morsom, kom
instruktoren i stedet til at copy-paste følgende ikke-komplette udsnit af en
rapport som vedkommende engang skrev i et projekt om en oversætter:

\begin{quote}
\subsubsection{If-then-else}

Eftersom oversætterens target ikke understøtter branches godt, er vi
interesserede i at fjerne dem så aggressivt som muligt fra den interne
repræsentation.  Internt er branches udtrykt som

\texttt{if} cond \texttt{then} exp$_1$ \texttt{else} exp$_2$

hvor cond er $0$ eller $1$, og returværdien er exp$_1$ hvis cond er $0$, og
exp$_2$ hvis cond er $1$.

Vi laver derfor denne transforma
\end{quote}

Bagefter faldt instruktoren i søvn og ramte med næsen Enter-knappen på sit
tastatur, så den ufuldendte passage blev sendt til alle de studerende på kurset.


\subsubsection{(a)}

Færdiggør instruktorens afsnit: Beskriv en transformation der tager et
if-then-else-udtryk og giver et udtryk der \emph{ikke} indeholder et
if-then-else-udtryk, men som returnerer det samme for alle mulige inddata.
Ingen udtryk har sideeffekter.  Skriv det i pseudo-SML.  \emph{Vink}: Husk at
\texttt{0 + var} er det samme som bare \texttt{var}, og at \texttt{0 * var} er
det samme som bare \texttt{0}.


\subsubsection{(b)}

Beskriv en algoritme der tager et udtryk og vurderer om det terminerer eller ej.
Mere præcist, så skal den returnere en af to værdier som beskriver disse
resultater:
\begin{itemize}
\item ``Dette udtryk terminerer.''
\item ``Dette udtryk terminerer måske, måske ikke.''
\end{itemize}

Tænk også over hvordan du kan bruge denne algoritme til at fuldende din løsning
i (a), for hvad er problemet med en naiv (omend umiddelbart korrekt)
transformation?


\subsection{Opgave 3}

% Uløst problem: Polignac's conjecture

Til en øvelsestime er dine medstuderende lige lovligt næsvise under en
gennemgang af noget vigtigt.  Din instruktor beslutter derfor at give jer en
\emph{sjov} opgave fra talteori-feltet, som I vel bare kan løse på 5 minutter!
Følgende konjektur skrives på tavlen:

\begin{quote}
  Lad $n$ være et vilkårligt, positivt, lige tal.  Der er nu uendeligt mange
  tilfælde hvor to fortløbende primtal $p_k$ og $p_{k+1}$ har forskellen $n$,
  altså at
\begin{align*}
p_{k+1} - p_k = n.
\end{align*}
\end{quote}

Bevis eller modbevis denne konjektur.

\vspace{.1in} \textbf{\emph{NB: Der udloddes en flaske snaps til den første som
    kommer op i baren med en korrekt besvarelse af denne opgave!}}



\setlength{\parindent}{0mm}
